%        File: notes.tex
%     Created: Thu Sep 24 12:00 PM 2020 M
% Last Change: Thu Sep 24 12:00 PM 2020 M
%
\documentclass[a4paper]{article}

\usepackage{amsfonts}

\newtheorem{definition}{Definition}
\newtheorem{theorem}{Theorem}

\newcommand{\C}{\mathbb{C}}
\newcommand{\R}{\mathbb{R}}

\usepackage{amsmath}

\begin{document}

\section{Cauchy's Theorem}

\subsection{Contour Integrals}

\subsubsection{Definitions}
\begin{definition}[Integral of a curve in $\C$]
  \[
    \int^b_a h(t) dt = \int^b_a u(t) dt + i \int^b_a v(t) dt
  \]
  Where \(h: [a, b] \subset \R \to \C \)


\end{definition}


\begin{definition}[continuous curve or contour in $\C$]
  A continuous map \(\gamma : [a, b] \to \C\).
\end{definition}

\begin{definition}[piecewise $C^1$ curve]
  A curve whose interval $[a, b]$ we can divide up $a = a_0 < a_1 <  \ldots < a_n = b$ such that the derivative $\gamma'(t)$ exists on each open subinterval $(a_i, a_{a + 1})$ and is continuous on $[a_i, a_{i + 1}]$, where continuous on $[a_i, a_{i + 1}]$ means $\lim_{t \to a_i^+} \gamma'(t)$ and $\lim_{t \to a_{i+1^-}} \gamma'(t)$ exist.
\end{definition}

\begin{definition}[Integral of $f$ along $\gamma$]
  \[
    \int_\gamma f = \int_\gamma f(z) dz = \sum^{n - 1}_{i = o} \int^{a+1}_{a_i} f(\gamma(t))\gamma'(t) dt
  \]
  Where $f$ is a continuous and defined on an open set $A \subset \C$ and that $\gamma: [a, b] \to \C$ is a piecewise smooth curve satisfying $\gamma([a, b]) \subset A$.\\

  \paragraph{Relation to line integrals in a vector field}
  The definition is analogous to the following definition of a line integral from vector calculus. Let $P(x, y)$ and $Q(x, y)$ be  real-valued functions from $x$ to $y$ and let $\gamma$ be a curve.
  \[
    \int_\gamma P(x, y) dx + Q(x, y) dy = \sum^{n- 1}_{i = 0} \int^{a_{i+1}}_{a_i} [P(x(t), y(t)) \frac{dx}{dt} + Q(x(t), y(t)) \frac{dy}{dt}] dt
  \]
  where $\gamma(t) = (x(t), y(t))$.
\end{definition}


\begin{definition}[Opposite curve $-\gamma$]
  \[
    - \gamma : [a , b] \to \C
  \]
  \paragraph{Relation to $\gamma$}
  $(- \gamma)(t) = \gamma(a + b - t)$.

  \paragraph{Geometric Intuition}
  $- \gamma $ is $- \gamma$ traversed in the opposite sense. 
\end{definition}

\begin{definition}[join or sum or union of $\gamma_1 + \gamma_2$ ]
  Suppose that $\gamma_1 : [a, b] \to \C$ and that $\gamma_2: [b, c] \to \C$ with $\gamma_1(b) = \gamma_2(b)$. 
  Define $\gamma_1 + \gamma_2 : [a, c] \to \C$ as 
  \[ (\gamma_1 + \gamma_2)(t) = \begin{cases} 
      \gamma_1(t) & \text{ if } t \in [a, b] \\ 
      \gamma_2(t) & \text{ if } t \in [b, c] \\ 
   \end{cases}
\]

  Where $\gamma_1$ and $\gamma_2$ are two curves. 
\end{definition}





\begin{definition}[Reparametrization]
  Let $\gamma : [a, b] \to \C$ be a piecewise smooth curve.
  A  piecewise smooth curve $\bar{\gamma}: [\bar{a}, \bar{b}] \to \C$ is called a reparametrization of $\gamma$ if there is some $C^1$ function $\alpha: [a, b] \to [\bar{a}, \bar{b}]$ with $\alpha'(a) = \bar{a}$ and $\alpha(b) = \bar{b}$ such that $\gamma(t) = \hat{\gamma}(\alpha(t))$
  \paragraph{Additional conditions}
  If $\gamma'(t) > 0 $ (hence increasing), $\gamma(a) = \bar{a}$, and $\alpha(b) = \bar{b}$ imply that $\bar{\gamma}$
  traverses the curve in the same sense that $\gamma$ does. This is sufficient and necessary to represent the same geometric curve.

  \paragraph{Non-differentiable points are maintained}
  The points in $[\bar{a}, \bar{b}]$ at which $\bar{\gamma}'$ does not exist correspond under $\alpha$ to the points of $[a, b]$ at which $\gamma'$ does not exist.
\end{definition}


\begin{definition}[Arc length of a curve ]
  \(\gamma : [a, b] \to \C\)
  \[
    l(\gamma) = \int_a^b|\gamma'(t)| dt = \int^b_a \sqrt{x'(t)^2 + y'(t)^2}dt
  \]

  \paragraph{Arc length is independent of parameterization}
  Geometrically similar to reparameterization. 

\end{definition}


\begin{definition}[Curve equalities and inequalities]
  Let $f$ be continuous on an open set $A$ and let $\gamma$ be a piecewise $C^1$ curve in $A$. If there is some constant $M \geq 0$ such that $|f(z)| \leq M$ for all points $z$ on $\gamma$ (that is, for all $z$ of the form $\gamma(t)$ for some $t$), then
  \[
    |\int_\gamma f| \leq M l(\gamma)
  \]

  \[
    |\int_\gamma f| \leq \int_\gamma |f| |dz|
  \]

  Where
  \[
    \int_\gamma |f| |dz| = \int_a^b |f(\gamma(t))|| \gamma'(t)| dt
  \]
\end{definition}
\paragraph{Usage}
Use this a s a basic tool to estimate the size of integrals. 






\begin{definition}[Conservative]
  A force field is called conservative if the net work done along a closed path is always $0$ or equivalently if the work done between two points is independent of the path taken between those points. 
\end{definition}

\begin{definition}[Potential Energy]
  If the force field is conservative, then the integral defines the potential energy, whose gradient is the original force field. 
\end{definition}

\subsection{Cauchy's Theorem - A First Look}
\begin{definition}[a simple closed curve]
  A curve is simple if it intersects itself only at its endpoints. 
\end{definition}




\begin{definition}[Simply Connected]
  A region \(A \subset \C\) is called simply connected if \(A\) is connect and every closed curve \(\gamma\) in \(A\) can be deformed in \(A\) to some constant curve \(\bar{\gamma(t)} = z_0 \in A\). 
  
  \paragraph{}{Geometrically}
  \begin{itemize}
    \item A region without holes. 
  \end{itemize}

\end{definition}

\begin{definition}[Homotopic to a point or contractable to a point]
  Curves inside of a simply connect set. 

  \paragraph{}{Geometrically}
  \begin{itemize}
    \item A curve that does loop around a hole cannot be shrunk down to a point in \(A\) without leaving \(A\). This curve is not homotopic to a point. 
  \end{itemize}
\end{definition}







\begin{definition}[Branch of the Logarithm function on \(A\)]
  Our choice of a \(F\) such that \(e^{F(z)} = z\).
\end{definition}


\subsection{A Closer Look at Cauchy's Theorem}












\begin{definition}[homotopic with fixed endpoints to \(\gamma_1\)]
  Suppose \(\gamma_0 : [0, 1] \to G \) and \(\gamma_1 : [0, 1] \to G \) are two continuous curves from \(z_0\) to \(z_1\) in a set \(G\). 
  We say that \(\gamma_0\) is homotopic with fixed endpoints to \(\gamma_1\) in \(G\) if there is a continuous function \(H : [0, 1] \times [0, 1] \to G\) from the unit square \([0, 1] \times [0, 1]\) into \(G\) such that 
  \begin{enumerate}
    \item \(H(0, t) = \gamma_0(t)\)  for \(0 \leq t \leq 1\). 
    \item \(H(1, t) = \gamma_1(t)\)  for \(0 \leq t \leq 1\). 
    \item \(H(s, 0) = z_0\) for \(0 \leq s \leq 1\). 
    \item \(H(s, 1) = z_1\)  for \(0 \leq s \leq 1\). 
  \end{enumerate}

  \paragraph{Note}
  If we fix \(\gamma_s(t) = H(s, t)\), then each \(\gamma_s\) is a continuous curve in \(G\). 

  \paragraph{Construction}
  One possible homotopy from one curve to the other is \(H(s, t) = t + t^{1 + s} i\).
   Another is \(H(s, t) = s(t + t^2 i) + (1 - s)(t + ti)\), which is the straight line between \(t + ti\) and \(t + t^2 i\)
\end{definition}

\begin{definition}[homotopic as closed curves in \(G\)]
  Suppose \(\gamma_0 : [0, 1] \to G \) and \(\gamma_1 : [0, 1] \to G \) are two continuous curves closed curves in a set \(G\). 
  We say that \(\gamma_0\) and \(\gamma_1\) are homotopic as closed curves in \(G\) if there is a continuous function \(h : [0, 1] \times [0, 1] \to G\) from the unit square \([0, 1] \times [0, 1]\) into \(G\) such that
  \begin{enumerate}
    \item \(H(0, t) = \gamma_0(t)\) for \(0 \leq t \leq 1\).
    \item \(H(1, t) = \gamma_1(t)\) for \(0 \leq t \leq 1\).
    \item \(H(s, 0) = H(s, 1)\) for \(0 \leq s \leq 1\).
  \end{enumerate}
  \paragraph{Note}
  If we fix \(\gamma_s(t) = H(s, t)\), then each \(\gamma_s\) is a continuous curve in \(G\). 

  \paragraph{Examples}
  The unit circle and ellipse are homotopic closed curves in an annulus. With one such \(H\) as \(H(s, t) = (1 + s) \cos t + i \sin t\). 
  If we would have used a disk instead of an annulus then both curves would have been homotopic to a point. Really an curve is homotopic to a point in a disk. 

  \paragraph{Usage}
  With more complicated regions, we often rely on our geometric intuition to determine when two curves are homotopic. 
  That is we try to decide whether we can continuously deform one curve to the other without leaving our region. 
  One reason is that we rarely use the homotopies \(H\) explicitly in practice; they are usually theoretical tools whose existence allows us to claim something else, such as equality of two integrals.
  In many situations homotopies might be quite complicated to write down.
  However, we must be prepared to justify our geometric intuition either with an explicit \(H\) or a proof of its existence in any particular situation. 


\end{definition}

\begin{definition}[Simply Connected]
  A connected set \(G\) is called simply connected if every closed curve \(\gamma\) in \(G\) is homotopic (as a closed curve) to a point in \(G\), that is, to some constant curve. 
\end{definition}


\begin{definition}[Convex Set]
  A set \(A\) is called convex if it contains the straight-line segment between every pair of its points. 
  That is, if \(z_0\) and \(z_1\) are in \(A\), then so is \(s z_1 + (s - 1) z_0\) for every number \(s\) between \(0\) and \(0\).
\end{definition}

\begin{definition}[Smooth homotopy]
  A homotopy \(H : [0, 1] \times [0, 1] \to G\) is called smooth if the intermediate curves \(\gamma_s(t)\) are piecewise \(C^1\) functions of \(t\) for each \(s\) and the cross curves \(\lambda_t(s)\) are piecewise \(C^1\) functions of \(s\) for each \(t\). 
\end{definition}

\section{Cauchy's Integral Formula}

\begin{definition}[Index of a curve]
Let \(\gamma \) be a closed curve in \(\C\) and \(z_0 \in \C\) be a point not on \(\gamma\). 
Then the index of \(\gamma\) with respect to \(z_0\) is defined by 
\[I(\gamma; z_0) = \frac{1}{2 \pi i} \int_\gamma \frac{dz}{z - z_0} \].

\paragraph{Also called}
\begin{itemize}
  \item Winding number of \(\gamma\)
  \item say \(\gamma \) winds around \(z_0\), \(I(\gamma, z_0) \) times. 
\end{itemize}

\paragraph{Used to}
Can be used to formalize inside and outside a curve. This can also be done by Jordan Curve Theorem (a topological result).

\paragraph{Note}
The index is a continuous function of \(z\) as long as \(z\) does not cross \(\gamma\)

\end{definition}



\end{document}



