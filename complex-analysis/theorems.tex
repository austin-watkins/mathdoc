%        File: notes.tex
%     Created: Thu Sep 24 12:00 PM 2020 M
% Last Change: Thu Sep 24 12:00 PM 2020 M
%
\documentclass[a4paper]{article}

\usepackage{amsfonts}

\newtheorem{theorem}{Theorem}

\newcommand{\C}{\mathbb{C}}
\newcommand{\R}{\mathbb{R}}

\usepackage{amsmath}

\begin{document}

\section{Cauchy's Theorem}

\subsection{Contour Integrals}

\subsubsection{Definitions}


\begin{theorem}[Integral of $f$ along $\gamma$ in terms of $u$ and $v$]
  \[
  \int_\gamma f = \int_\gamma [u(x, y) dx - v(x, y)dy] + i \int_\gamma [u(x, y)dy + v(x, y)dx]
  \]
  \paragraph{How to remember}
  \[f(z) dz = (u + iv)(dx + i dy) = u dx - v dy + i(v dx + u dy)\]

\end{theorem}




\begin{theorem}[Integration properties of curves]
  \[
    \int_\gamma (c_1 f + c_2 g) = c_1 \int_\gamma f + c_2 \int_\gamma g
  \]

  \[
    \int_{-\gamma} f = - \int_\gamma f
  \]

  \[
    \int_{\gamma_1 + \gamma_2} f = \int_{\gamma_1} f + \int_{\gamma_2} f 
  \]

  Where $f, g$ are continuous functions, $c_1, c_2$ are complex constants, and $\gamma, \gamma_1, \gamma_2$ piecewise $C^1$ curves. 
\end{theorem}



\begin{theorem}[The Integral of $f$ is maintained over reparameterizations]
  If $\bar{\gamma}$  is a reparametrization  of $\gamma$, then 
  \[
    \int_\gamma f = \int_{\bar{\gamma}} f
  \]
  for any continuous $f$ defined on an open set contining the image of $\gamma = $ image of $\bar{\gamma}$.
\end{theorem}



\begin{theorem}[Fundamental Theorem of Calculus - Contour Integrals]
  Suppose that $\gamma : [0, 1] \to \C$ is a piecewise smooth curve and that $F$ is a function defined and analytic on an open set $G$ containing $\gamma$. Assume $F'$ is continuous (later we find this is redundant). Then 
  \[
    \int_\gamma F'(z) dz = F(\gamma(1)) - F(\gamma(0)).
  \]
  In particular, if $\gamma(0) - \gamma(1)$, then 
  \[
    \int F'(z)dz = 0.
  \]
\end{theorem}

\begin{theorem}[Connection between zero derivative and constant]
If $f$ is a function defined and analytic on an open connected set $G \subset \C$, and if $f'(z) = 0$ for every point $z$ in $G$, then $f$ is constant on $G$. 
\end{theorem}


\begin{theorem}[Path Independence Theorem]
  Suppose $f$ is a continuous function on an open connected set $G \subset \C$.
  Then the following are equivalent: 
  \begin{itemize}
    \item Integrals are path-independent: If $z_0$ and $z_1$ are any two points in $G$ and $\gamma_0$ and $\gamma_1$ are paths in $G$ from $z_0$ to $z_1$, then 
      \[
        \int_{\gamma_0} f(z) dz = \int_{\gamma_2} f(z) dz.
      \]
    \item Integrals around closed curves are 0: If $\Gamma$ is a closed curve (loop) laying in $G$, then $\int_\Gamma f(z) dz = 0$.
    \item There is a (global) antiderivatve for $f$ on $G$: there is a function $F$ defined and analytic on all of $G$ such that $F'(z) = f(z)$ for all $z$ in $G$. 
  \end{itemize}
\end{theorem}

\begin{theorem}[Green's Theorem]
  For continuously differentiable functions \(P(x, y)\) and \(Q(x, y)\), 
  \[
    \int_\gamma P(x, y) dx + Q(x, y) dy = \int \int_A [\frac{\partial Q}{\partial x}(x, y) - \frac{\partial P}{\partial y}(x, y)] dx dy
    \]
  \paragraph{Used for}
    \begin{itemize}
      \item Making precise notions like the inside of curves clear. 
    \end{itemize}
\end{theorem}

\begin{theorem}[Preliminary Version of Cauchy's Theorem]
  Suppose that \(f\) is analytic, with derivative \(f\) continuous on and inside a simple closed curve \(\gamma\). 
  Then
  \[
    \int_\gamma f = 0
    \]
\end{theorem}

\begin{theorem}[Preliminary Version of the Deformation Theorem]
  Let \(f\) be analytic on a region \(A\) and let \(\gamma\) be a simple closed curve in \(A\). 
  Suppose that \(\gamma\) can be continuously deformed to another simple closed curve \(\bar{}\gamma\) without passing outside of the region \(A\) Then
  \[
    \int_\gamma f = \int_{\bar{\gamma}} f.
  \]
  \paragraph{Used for}
    \begin{itemize}
      \item To study curves over functions with a singularity inside of the curves.
       We can deform the curve to one that might be easier to evaluate. 
    \end{itemize}
\end{theorem}



\begin{theorem}[Cauchy's Theorem for a Simply Connected Region]
  If \(f\) is analytic on a simply connected region \(G\) and \(\gamma\) is a closed curved in \(G\), then 
  \[
    \int_\gamma f = 0.
    \]
\end{theorem}

\begin{theorem}[Path Independence on Simply Connected Regions]
  Suppose that \(f\) is analytic on a simply connected region \(A\). 
  Then for any two curves \(\gamma_1\) and \(\gamma_2\) joining two points \(z_0\) and \(z_1\) in A, we have
  \[ \int_{\gamma_1} = \int_{\gamma_2} f\]
\end{theorem}

\begin{theorem}[Antiderviative Theorem]
  Let \(f\) be a function defined on a simply connected region \(A\). 
  Then there is an analytic function \(F\) defined on \(A\) that is unique up to an additive constant, such that \(F'(Z)  = f(z)\) for all \(z\) in \(A\). 
  We call \(F\) the antiderivative of \(f\) on \(A\). 
\end{theorem}


\begin{theorem}[Existence of Logarithms]
  Let \(A\) be a simply connected region and assume that \(0 \not \in A\).
  Then there is an analytic function \(F(z)\), unique up to the addition of multiples of \(2 \pi i\), such that \(e^{F(z)} = z\).
\end{theorem}



\subsection{A Closer Look at Cauchy's Theorem}

\begin{theorem}[Cauchy's Theorem for a Rectangle]
  Suppose that \(R\) is a rectangular path with sides parallel to the axes and that \(f\)  is a function defined and analytic on an open an open set \(G\) containing \(R\) and its interior.
  Then 
  \[\int_R f = 0.\]

\end{theorem}

\begin{theorem}[Cauchy's Theorem for a Disk]
  Suppose that \(f : D \to \C\) is analytic on a disk \(D = D(z_0; \rho) \subset \C\).
  Then
  \begin{enumerate}
    \item \(f\) has an antiderivative on \(D\); that is there is a function \(F : D \to C\) that is analytic on \(D\) and that satisfies \(F'(z) = f(z)\) for all \(z\) in \(D\).
    \item If \(\Gamma\) is any closed curve in \(D\), then \(\int_\Gamma f = 0\).
  \end{enumerate}

\end{theorem}

\begin{theorem}[Cauchy's Theorem for Deleted Neighborhoods]
  Suppose that \(R\) is a rectangular path with sides parallel to the axes, that \(f\)is a function defined on an open set  \(G\) containing \(R\) and its interior, and that \(f\) is analytic on \(G\) except at some fixed point \(z_1\) in \(G\) which is not on the path \(R\). 
  Suppose that at \(z_1\), the function \(f\) satisfies \(\lim_{z \to z_1} (z - z_1) f(z) = 0 \). 
  Then \(\int_R f = 0\).

  \paragraph{Note}
  The limit condition holds under any of the following conditions.
  \begin{itemize}
    \item If \(f\) is bounded in a deleted neighborhood of \(z_1\)
    \item If \(f\) is continuous on  \(G\)
    \item If \(\lim_{z \to z_1} f(z)\) exists. 
  \end{itemize}
\end{theorem}

\begin{theorem}
  Suppose that \(R\) is a rectangular path with sides parallel to the axes, that \(f\) is a function defined and continuous on an open set \(G\) containing \(R\) and its interior, and that \(f\) is analytic on \(G\) except at some fixed point \(z_1\) in \(G\). 
  Then \(\int_R f = 0\).
\end{theorem}






\begin{theorem}[Strengthened Cauchy's Theorem for a Disk]
  Suppose that \(f : D \to \C\) is continuous on \(D\) and analytic on \(D \setminus \{z_1\}\) for some fixed \(z_1\) in \(D\). 
  Then
  \begin{enumerate}
    \item \(f\) has an antiderivative on \(D\); that is there is a function \(F : D \to C\) that is analytic on \(D\) and that satisfies \(F'(z) = f(z)\) for all \(z\) in \(D\).
    \item If \(\Gamma\) is any closed curve in \(D\), then \(\int_\Gamma f = 0\).
  \end{enumerate}
\end{theorem}







\begin{theorem}
  If \(A\) is a convex region, then any two closed curves in \(A\) are homotopic as closed curves in \(A\), and any two curves with the same endpoints are homotopic with fixed endpoints. 
\end{theorem}

\begin{theorem}[A convex region is simply connected].
\end{theorem}

\begin{theorem}[Deformation Theorem]
  Suppose that \(f\) is an analytic function on an open set \(G\) and that \(\gamma_0\) and \(\gamma_1\) are piecewise \(C^1\) curves in \(G\). 
  \begin{enumerate}
    \item If \(\gamma_0 \) and \(\gamma_1\) are paths from \(z_0\) to \(z_1\) and are homotopic in \(G\) with fixed endpoints, then 
      \[\int_{\gamma_0} f = \int_{\gamma_1} f.\]
    \item If \(\gamma_0\) and \(\gamma_1\) are closed curves with are homotopic as closed curves in \(G\), then 
      \[\int_{\gamma_0} f = \int_{\gamma_1} f.\]
  \end{enumerate}

  \paragraph{Note}
  Technically the proof given in the book requires the homotopies to be smooth, but there is a proof without this restriction. 
\end{theorem}


\begin{theorem}[Homotopy Form of Cauchy's Theorem]
  Let \(f\) be analytic on a region \(G\). 
  Let \(\gamma\) be a closed curve in \(G\) which is homotopic to a point in \(G\). Then
  \[\int_\gamma f = 0.\]
  
\end{theorem}






\end{document}



